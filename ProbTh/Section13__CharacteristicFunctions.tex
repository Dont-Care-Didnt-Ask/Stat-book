\section{Характеристические функции и их свойства}
\begin{defn}
    \textit{Характеристическая функция случайной величины} $\xi$~--- функция $\varphi_{\xi} \colon \mathbb{R} \rightarrow \mathbb{C}$:
    \begin{equation*}
        \varphi_{\xi}(t)
        = \mathbb{E} e^{it \xi}
        = \mathbb{E} \cos (t \xi)+i \mathbb{E} \sin (t \xi) = \int\limits_{\mathbb{R}}^{} e^{i t x} d F_{\xi}(x),
    \end{equation*}
    где интеграл справа называется \textit{интегралом Фурье-Стильтьеса}.
    
    Для абсолютно непрерывного распределения характеристическая функция имеет вид
    \begin{equation*}
        \varphi_{\xi}(t)=\int\limits_{\mathbb{R}} e^{i t x} f(x) \, dx.
    \end{equation*}
    Для дискретного, соответственно
    \begin{equation*}
        \varphi_{\xi}(t)=\sum\limits_{i} e^{i t x_{i}} \, \mathbb{P}\left(\xi=x_{i}\right) \! .
    \end{equation*}
\end{defn}

\begin{exmp}
    Характеристическая функция стандартной нормальной случайной величины $\xi \sim \mathbf{N}(0;1)$:
    \begin{multline*}
        \varphi_{\xi}(t) 
        = \frac{1}{\sqrt{2 \pi}} \int\limits_{-\infty}^{\infty} e^{i t x} e^{-x^{2} / 2} d x
        = \frac{1}{\sqrt{2 \pi}} \int\limits_{-\infty}^{\infty} e^{-t^{2} / 2} e^{-x^2/2 \,+\, itx \,+\, t^2/2} d x = \\
        = \frac{1}{\sqrt{2 \pi}} \int\limits_{-\infty}^{\infty} e^{-t^{2} / 2} e^{-(x-i t)^{2} / 2} d x =  
         e^{-t^{2} / 2} \frac{1}{\sqrt{2 \pi}} \int\limits_{-\infty}^{\infty} e^{-(x-i t)^{2} / 2} d(x-i t)
        = e^{-t^{2} / 2}.
    \end{multline*}
\end{exmp}

\begin{namedthm}[Свойства характеристической функции]\leavevmode
    \begin{enumerate}
        \item 
            Характеристическая функция существует для любой случайной величины $\xi$.
        \item 
            $\forall \, \xi,~ \forall \, a, b \in \mathbb{R} \colon \varphi_{a \xi + b}(t) = e^{itb} \varphi_{\xi}(at)$.
        \item 
            \begin{enumerate}
                \item $ \bigl| \varphi_{\xi}(t) \bigr| = \bigl| \mathbb{E} e^{i t \xi} \bigr| \leqslant 1 $;
                \item $ \varphi_{\xi}(0) = 1 $;
                \item $ \overline{\varphi_{\xi}(t)} = \varphi_{\xi}(-t) = \varphi_{-\xi}(t) \quad \forall \, t \in \mathbb{R}$.
            \end{enumerate}
            \textbf{Следствие (из c).}
                Если характеристическая функция вещественнозначна, то она является чётной.
        
        \item 
            Если случайные величины $\xi$ и $\eta$ независимы, то
            \begin{equation*}
                \varphi_{\xi + \eta}(t) = \varphi_{\xi}(t) \varphi_{\eta}(t).
            \end{equation*}
        \item 
            Характеристическая функция равномерно непрерывна.
        \item 
            Если существует абсолютный момент $k$-го порядка $\mathbb{E}|\xi|^{k} < \infty,~ k \geqslant 1$, 
            то существует непрерывная $k$-я производная характеристической функции:
            \begin{equation*}
                \left.\cfrac{\partial^{k}}{\partial t^{k}} \, \varphi_{\xi}(t)\right|_{t=0}= i^{k} \, \mathbb{E} \xi^{k}.
            \end{equation*}
            
            Если существует непрерывная производная характеристической функции порядка $k = 2n, n \in \mathbb{N}$, 
            то существует абсолютный момент порядка $k = 2n: \; \mathbb{E}|\xi|^k = \mathbb{E}\xi^k$ (а следовательно, и все предыдущие) 
            и его можно вычислить по той же формуле.
        
        \item 
            Характеристическая функция случайно величины $\xi$ однозначно определяет её функцию распределения $F_{\xi}(x)$. 
            Ряд распределения или плотность восстанавливаются по характеристической функции с помощью преобразования Фурье.
            
            Дискретное распределение:
            \begin{equation*}
                \mathbb{P}(\xi=k)=\frac{1}{2 \pi} \int\limits_{-\pi}^{\pi} e^{-i t k} \varphi_{\xi}(t) \, d t, k \in \mathbb{Z}.
            \end{equation*}
            
            Абсолютно непрерывное распределение:
            \begin{equation*}
                f_{\xi}(x)=\frac{1}{2 \pi} \int\limits_{-\infty}^{\infty} e^{-i t x} \varphi_{\xi}(t) \, d t, \quad x \in \mathbb{R}.
            \end{equation*}
        \item 
            $\xi_n \xrightarrow[n \to \infty]{\text{d}} \xi \quad \Leftrightarrow \quad \varphi_{\xi_{n}}(t) \xrightarrow[n \to \infty]{} \varphi_{\xi}(t)$ (теорема Леви о непрерывном соответствии).
    \end{enumerate}
\end{namedthm}

\begin{proof}
    \begin{enumerate}
        \item 
            Существование характеристической функции равносильно равномерной сходимости соответствующего интеграла. 
            Докажем её по признаку Вейерштрасса:
            \begin{equation*}
                \left|\varphi_{\xi}(t)\right|=\left|\int\limits_{\mathbb{R}} e^{i t x} d F(x)\right| 
                \leqslant \int\limits_{\mathbb{R}}\left|e^{i t x}\right| d F(x)=\int\limits_{\mathbb{R}} d F(x)=1.
            \end{equation*}
        \item 
            $\varphi_{a \xi+b}(t) 
            = \mathbb{E}e^{i t(a \xi+b)}
            = e^{i t b} \mathbb{E}e^{i t a \xi}
            = e^{i t b} \varphi_{\xi}(t a)$.
        \item 
            Неравенство доказано в пункте 1, равенство $(b)$ очевидно.
            \begin{equation*}
                \varphi_{\xi}(-t) = \mathbb{E}cos(-t \xi) + i\mathbb{E}sin(-t \xi) 
                = \mathbb{E}cos(t \xi) - i\mathbb{E}sin(t \xi) = \overline{\varphi_{\xi}(t)}.
            \end{equation*}
            Оставшиеся равенства следуют из второго свойства.
        \item 
            $\varphi_{\xi + \eta}(t) 
            = \mathbb{E}e^{it(\xi + \eta)} 
            = \text{\{независимость\}}
            = \mathbb{E}e^{it\xi}\mathbb{E}e^{it\eta}
            = \varphi_{\xi}(t)\varphi_{\eta}(t).$
        \item 
            Выберем сколь угодно малое $\varepsilon > 0$ и оценим разность значений характеристической функции в точках $t$ и $t + h$:
            \begin{multline*}
                \bigl| \varphi(t+h)-\varphi(t) \bigr| 
                = \left|\int\limits_{\mathbb{R}} \left(e^{i(t+h) x}-e^{i tx}\right) d F(x)\right|
                = \left|\int\limits_{\mathbb{R}} e^{i t x}\left(e^{i h x}-1\right) d F(x)\right| \leqslant \\
                \leqslant \int\limits_{\mathbb{R}} \left|e^{i h x}-1\right| d F(x)=\int\limits_{|x| \leqslant R}\left|e^{i h x}-1\right| d F(x)+\int\limits_{|x|>R}\left|e^{i h x}-1\right| d F(x)
            \end{multline*}
            Теперь выберем $R$ настолько большим, чтобы $\mathbb{P}\bigl(|x|>R\bigr) < \frac{\varepsilon}{4}$. 
            Поскольку $\left|e^{i h x}-1\right| \leqslant 2$, второй интеграл при этом не превосходит по величине~$\frac{\varepsilon}{2}$. 
            После этого выберем $h$ столь малым, чтобы $\left|e^{i h x}-1\right|<\frac{\varepsilon}{2}~$ при всех $|x| \leqslant R$. 
            Тогда и первый интеграл не превосходит $\frac{\varepsilon}{2}$ и, таким образом, по заданному $\varepsilon > 0$ подобрано столь малое $h >0$, что ${|\varphi(t+h)-\varphi(t)|<\varepsilon~ \forall \, t \in \mathbb{R}}$.
        \item 
            Если существует $\mathbb{E}\xi^{k}<\infty,~ k \geqslant 1$, то для всех $m = \overline{1, k}$ существуют $\mathbb{E}\xi^{m}<\infty$. Следовательно,
            \begin{equation*}
                \left|\int\limits_{\mathbb{R}}(i x)^{m} e^{i t x} d F(x)\right| \leqslant \int\limits_{\mathbb{R}}|x|^{k} d F(x)=\mathbb{E}|\xi|^{m}<\infty \quad \forall \, m = \overline{1, k}
            \end{equation*}
            Т.е. интегралы $\int\limits_{\mathbb{R}}(i x)^{m} e^{i t x} d F(x)$ сходятся равномерно по $t$, а значит, дифференцирование по $t$ можно менять местами с операцией интегрирования, откуда
            \begin{equation*}
                \varphi_{\xi}^{(m)}(t)=i^{m} \int\limits_{\mathbb{R}} x^{m} e^{i t x} d F(x),~ \varphi_{\xi}^{(m)}(0)=i^{m} \int\limits_{\mathbb{R}} x^{m} d F(x)=i^{m} \mathbb{E}\xi^{m}
            \end{equation*}
            
            Пусть у характеристической функции существует непрерывная производная чётного порядка $k$. 
            Характеристическая функция и её производные непрерывны, функция $e^{itx}$ бесконечно (а значит, и нужные нам $k$ раз) дифференцируема по $t$, 
            и можно показать, что при этих условиях можно поменять знаки интегрирования и дифференцирования местами.
            %\footnote{Достаточно обозначить интеграл $\int\limits_{\mathbb{R}}{x^{m} e^{itx}dF(x)}, \, m = \overline{0, k-1}$ за $G(t)$ 
            %и расписать его производную по $t$ по определению, воспользовавшись тем, что $\int\limits_{\mathbb{R}} o(\Delta t) dF(x) = o(\Delta t)$.}
            Тогда мы получим, что 
            $$ \varphi^{(k)}(0) = i^k \left.\mathbb{E}\xi^k e^{itx}\right|_{t=0} = i^k \mathbb{E}\xi^k = 
            \int\limits_{\mathbb{R}} x^k dF(x) $$
            В силу чётности $k ~ x^k \geqslant 0$, а значит, указанный интеграл сходится абсолютно, что и означает существование искомого математического ожидания.
    \end{enumerate}
\end{proof}

\pagebreak

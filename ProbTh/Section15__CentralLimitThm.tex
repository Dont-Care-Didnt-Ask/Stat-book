\section{Центральная предельная теорема}
\begin{namedthm}[Центральная предельная теорема]
    Пусть $\xi_{1}, \xi_{2}, \ldots$~--- последовательность независимых одинаково распределенных невырожденных\footnote{Т.е. их дисперсия отлична от нуля. Важное требование, т.к. нам нужно делить на дисперсию.} случайных величин с $\mathbb{E} \xi_{1}^{2}<\infty$ и $S_{n}=\xi_{1}+\ldots+\xi_{n}$. 
    Тогда
    \begin{equation*}
        \mathbb{P}\left(\frac{S_{n}-\mathbb{E} S_{n}}{\sqrt{\mathbb{D} S_{n}}} \leqslant x\right)
        \xrightarrow[n \to +\infty]{}
        \Phi(x) = \frac{1}{\sqrt{2 \pi}} \int\limits_{-\infty}^{x} e^{-\frac{u^{2}}{2}} du~~ \forall \, x \in \mathbb{R}
    \end{equation*}
\end{namedthm}
\begin{proof}
Пусть $\mathbb{E} \xi_{1}=m,\, \mathbb{D} \xi_{1}=\sigma^{2}$. 
Введём $X = \xi_1 - m$ и $\varphi_X(t)=\mathbb{E} e^{i tX}$. 
Введём также
\begin{equation*}
    \varphi_{n}(t)=\mathbb{E} e^{i t \frac{S_{n}-\mathbb{E} S_{n}}{\sqrt{\mathbb{D} S_{n}}}} = 
    \left[\varphi_X\left(\frac{t}{\sigma \sqrt{n}}\right)\right]^{n}
\end{equation*}

В силу разложения характеристической функции (при существовании соответствующих моментов)
\begin{equation*}
    \varphi_{X}(t)=1+i t \mathbb{E} X+\ldots+\frac{(i t)^{n}}{n !} \mathbb{E} X^{n}+R_{n}(t)
\end{equation*}

Учитывая то, что $\mathbb{E}X = \mathbb{E}\left[ \xi_1 - m\right] = 0$, при $n=2$ получим 
\begin{equation*}
    \varphi(t)=1-\frac{\sigma^{2} t^{2}}{2}+\overline{o}\left(t^{2}\right), \quad t \rightarrow 0
\end{equation*}

Следовательно, для любого $t \in \mathbb{R}$ при $n \to +\infty$
\begin{equation*}
    \varphi_{n}(t)=\left[1-\frac{\sigma^{2} t^{2}}{2 \sigma^{2} n}+\overline{o}\left(\frac{1}{n}\right)\right]^n \rightarrow e^{-\frac{t^{2}}{2}}
\end{equation*}

Функция $e^{-\frac{t^{2}}{2}}$ является характеристической функцией $\mathbf{N}(0,1)$. 
В силу теоремы о непрерывном соответствии между функциями распределения и характеристическими функциями центральная предельная теорема доказана.
\end{proof}

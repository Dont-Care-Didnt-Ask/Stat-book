\section{Случайная величина. Порождённое и индуцированное вероятностные пространства. Функция распределения, ее свойства}

\begin{defn}
    \textit{$\text{Борелевская~} \sigma \text{-алгебра~} \mathfrak{B}$}~--- $\sigma \text{-алгебра}$, \hyperlink{generated_sigma}{порождённая} множеством всех открытых интервалов на $\mathbb{R}$ (иными словами, минимальная $\sigma$-алгебра, содержащая все открытые интервалы). 
    Элемент $B \in \mathfrak{B}$~--- \textit{борелевское множество}.
\end{defn}

\begin{defn}
    \textit{Борелевская функция}~--- функция $f\colon \mathbb{R} \mapsto \mathbb{R}$:
    \begin{equation*}
        \forall B \in \mathfrak{B} \quad f^{-1}(B) \in \mathfrak{B}
    \end{equation*}
    
    Т.е. борелевская функция~--- это функция, для которой прообраз (множество $f^{-1}(B) = \{x \colon f(x) \in B\}$) любого борелевского множества также является борелевским множеством.
\end{defn}

\begin{exmp}
    Функция Дирихле $D\colon \mathbb{R} \rightarrow \{0,1\}$
    \begin{equation*}
        D(x) =
        \begin{cases}
            1, & x \in \mathbb{Q}; \\
            0, & x \in \mathbb{R} \setminus \mathbb{Q}
        \end{cases}
    \end{equation*}
является борелевской. 

В самом деле, прообразом любого борелевского множества ${A, \text{ такого что } 1 \in A, \, 0 \notin A}$ является множество рациональных чисел; 
прообразом борелевского множества ${B \colon 0 \in B, \, 1 \notin B}$~--- множество иррациональных чисел; 
прооборазом борелевского множества ${C \colon 0 \in C, \, 1 \in C}$~--- вся вещественная прямая, 
наконец, прообразом борелевского множества ${E \colon 0 \notin E, \, 1 \notin E}$~--- пустое множество. 
Но $\mathbb{Q}, \, \mathbb{I}, \, \mathbb{R}, \, \varnothing $~--- борелевские множества, а значит, выполняется определение борелевской функции.
\end{exmp}


\begin{defn}
    Пусть $\mathcal{F}$~--- $\sigma$-алгебра на некотором непустом множестве $\Omega$.
    Функция $\xi$: $\Omega \mapsto \mathbb{R}$ называется \textit{измеримой относительно $\sigma$-алгебры $\mathcal{F}$}, если полный прообраз борелевского множества $B$ лежит в $\mathcal{F}$, т.е. 
    \begin{equation*}
        \xi^{-1}(B) = \{\omega \colon \xi(\omega) \in B \} \in \mathcal{F} \quad \forall B \in \mathfrak{B}.
    \end{equation*}
\end{defn}

\begin{rmrk}
    %Борелевская функция~--- это функция, измеримая относительно борелевской ${\sigma \text{-алгебры}}$.
    Если в определении измеримой функции положить $\Omega = \mathbb{R}$ и выбрать борелевскую $\sigma$-алгебру $\mathfrak{B}$, то мы получим в точности определение борелевской функции. 
    Т.е. борелевские функции~--- это подмножество измеримых функций.
\end{rmrk}

\subsubsection{Случайные величины}
\begin{defn}
    Пара $(X,\, \mathcal{U})$, где $X$~-- произвольное множество, а \\ $\mathcal{U}$~-- $\sigma$-алгебра над ним~--- \textit{измеримое пространство}. 
    Например, $(\Omega, \mathcal{F})$ и $(\mathbb{R}, \mathfrak{B})$~--- измеримые пространства. 
    Элементы $\sigma$-алгебры $\mathcal{U}$ называются \textit{измеримыми множествами}.
\end{defn}

\begin{defn}
    Пусть даны измеримые пространства $(\Omega, \mathcal{F})$ и $(\mathbb{R}, \mathfrak{B})$. 
    Тогда измеримая относительно $\mathcal{F}$ функция $\xi \colon \Omega \to \mathbb{R}$ называется \textit{случайной величиной}.
\end{defn}
\begin{rmrk}
    Если мы вспомним, что элементы $\sigma$-алгебры $\mathcal{F}$ называются событиями, то определение можно переформулировать следующим образом: 
    
    Пусть даны измеримые пространства $(\Omega, \mathcal{F})$ и $(\mathbb{R}, \mathfrak{B})$. 
    Функция $\xi \colon \Omega \mapsto \mathbb{R}$ называется \textit{случайной величиной}, если прообраз любого борелевского множества $B \in \mathfrak{B}$ является событием.
\end{rmrk}
\begin{exmp} Пусть дана функция $\xi$:
\begin{equation*}
    \xi(\omega) = 
    \begin{cases}
        1, & \omega \in \left[0, \frac{1}{2}\right]; \\
        0, & \omega \in \left(\frac{1}{2}, 1\right],
    \end{cases}
\end{equation*}

$\Omega = [0, 1], \mathcal{F} = \{\varnothing, \Omega\}$~--- минимальная ${\sigma \text{-алгебра}}$.  

Докажем неизмеримость функции $\xi$; для этого достаточно найти такое борелевское множество, прообраз которого не будет принадлежать ${\sigma \text{-алгебре}}$. 
В данном случае $\mathcal{F}$ состоит всего лишь из двух множеств~--- $\{[0, 1], \varnothing\}$.

Как и в примере с функцией Дирихле, попробуем перебрать борелевские множества, содержащие значения $\xi(\omega)$. 
Тогда мы увидим, что для любого борелевского множества $A \colon 0 \in A, \, 1 \notin A$~--- например, множества ${A_1 = \left(-\infty, \frac{1}{3}\right)}$~--- его прообразом является множество $\left(\frac{1}{2}, 1\right]$. 
Но это множество не входит в $\mathcal{F}$, а значит, $\xi(\omega)$ неизмерима относительно $\mathcal{F}$.

Отсюда можно сделать несколько выводов. 
Во-первых, измеримость функции зависит от выбора ${\sigma \text{-алгебры}}$ (как и было подчёркнуто в определении). 
Например, если мы рассмотрим ту же функцию $\xi(\omega)$ на том же $\Omega = [0, 1]$, 
но с другой $\sigma$-алгеброй ${\widehat{\mathcal{F}} = \{[0, 1], \left[0; \frac{1}{2}\right], \left(\frac{1}{2}, 1\right], \varnothing\}}$, 
то наша функция будет измеримой, а следовательно~--- случайной величиной (проверьте!).

Во-вторых (забегая немного вперёд), именно из-за неизмеримости $\xi(\omega)$ относительно $\mathcal{F}$ мы не можем посчитать вероятность попадания значений этой функции в некоторые интервалы, к примеру, $\mathbb{P}\left({\xi < \frac{1}{3}}\right)$. 
Ведь $\mathbb{P}\left( \xi < \frac{1}{3} \right) = \mathbb{P}\left(\xi \in A_1\right) = \mathbb{P}\left(\omega \in \left(\frac{1}{2}, 1\right] \right)$, но множество $\left(\frac{1}{2}, 1\right] \notin \mathcal{F}$, 
а вероятность~--- это отображение $\mathbb{P}: \mathcal{F} \mapsto \mathbb{R}$, и она не определена для этого множества.
\end{exmp} 

\begin{thm*}
    Пусть $\xi$~--- случайная величина, $g: \mathbb{R} \rightarrow \mathbb{R}$~--- борелевская функция. 
    Тогда $g(\xi)$~--- случайная величина.
\end{thm*}

\begin{proof}
    Напомним, что функция является случайной величиной, если прообраз любого борелевского множества принадлежит $\sigma$-алгебре, то есть
    $$\xi^{-1}(B) = \{\omega \colon \xi(\omega) \in B \} \in \mathcal{F} \qquad \forall B \in \mathfrak{B}.$$
    Рассмотрим прообраз произвольного борелевского множества для $\eta = g(\xi)$:
    $$\eta^{-1}(B) = \{\omega \colon g(\xi(\omega)) \in B) \} = \{\omega \colon \xi(\omega) \in g^{-1}(B)\}.$$
    Функция $g$ по предположению борелевская, следовательно, прообраз борелевского множества тоже будет борелевским: $g^{-1}(B) = C \in \mathfrak{B}$. 
    В свою очередь, $\xi$~--- случайная величина, и прообраз борелевского множества $C$ лежит в $\sigma$-алгебре $\mathcal{F}$. 
    Таким образом, 
    $$ \eta^{-1}(B) = \{\omega \colon g(\xi(\omega)) \in B) \} = \{\omega \colon \xi(\omega) \in C) \} \in \mathcal{F}.$$
    Мы получили, что прообораз произвольного борелевского множества принадлежит $\sigma$-алгебре. 
    Значит, $\eta = g(\xi)$~--- случайная величина.
\end{proof}

\begin{thm*}
    Пусть $\mathcal{E}$~--- класс подмножеств $\mathbb{R}$, $\sigma(\mathcal{E}) = \mathfrak{B}$ (например, $\mathcal{E}$~--- класс интервалов).

    Тогда $\xi$~--- случайная величина\ \ $\Leftrightarrow$\ \ $\forall E \in \mathcal{E}\colon ~\xi^{-1}(E) \in \mathcal{F}$.
\end{thm*}

\begin{proof}
    \begin{itemize}
        \item[$\Leftarrow$] Пусть $\mathcal{D} = \{D \colon D \in \mathfrak{B}, \, \xi^{-1}(D) \in \mathcal{F} \}$. Тогда $\mathcal{E} \subseteq \mathcal{D}$. Далее, в силу свойств прообразов\footnote{Читатель может сам легко проверить эти свойства, вспомнив определение прообраза: ${\xi^{-1}(B) = \{\omega \in \Omega \colon \xi(\omega) \in B\}}$.} и случайной величины $\xi$:
    \begin{gather*}
        \xi^{-1}\left(\bigcup\limits_i A_i\right) 
        = \bigcup\limits_i \xi^{-1}(A_i), \quad
        \xi^{-1}(\overline{A}) 
        = \overline{\xi^{-1}(A)}, \\
        \xi^{-1}\left(\bigcap\limits_i A_i\right) = \bigcap\limits_i \xi^{-1}(A_i) \\
        \text{для любых } A_i \in \mathcal{D}.
    \end{gather*}
    % что за A_i??
    % Ответ: A_i – это произвольные множества, принадлежащие классу D.
    Следовательно, $\mathcal{D}$~--- ${\sigma \text{-алгебра}}$. $\mathfrak{B} = \sigma(\mathcal{E}) \subseteq \sigma(\mathcal{D}) = \mathcal{D} \subseteq \mathfrak{B} \Rightarrow \mathfrak{B} = \mathcal{D}.$
    
    \item[$\Rightarrow$] Следует непосредственно из определения случайной величины, т.к. ${\mathcal{E} \subseteq \mathfrak{B}}$.
    \end{itemize}
\end{proof}

\begin{crlr}
    $\xi$~--- случайная величина $\Leftrightarrow$ $\forall x \in \mathbb{R} \colon \{\omega \colon \xi(\omega) < x \} \in \mathcal{F}$. 
    Причем вместо знака $<$ может стоять любой другой знак неравенства, как строгого, так и нестрогого.
\end{crlr}

\subsubsection{Порождённое и индуцированное вероятностные пространства}
\begin{defn}
    $\sigma\text{-алгебра}$, порожденная случайной величиной $\xi$:
    \begin{equation*}
        \mathcal{F}_\xi = \{\xi^{-1}(B), \, B \in \mathfrak{B} \}.
    \end{equation*}
\end{defn}

Отметим следующие факты:
\begin{enumerate}
    \item $\mathcal{F}_\xi \subset \mathcal{F}.$
    \item $\mathcal{F}_\xi$~--- ${\sigma \text{-алгебра}}$. Действительно:
    \begin{equation*}
        \xi^{-1}(\overline{B}) = \overline{\xi^{-1}(B)}, \quad
        \xi^{-1}\left(\, \bigcup\limits_{i=1}^{\infty}B_i \right) = \bigcup\limits_{i=1}^\infty \xi^{-1}(B_i),
    \end{equation*}
   если $B_i$ попарно не пересекаются.
\end{enumerate}

\begin{defn}
    Вероятностное пространство $(\Omega,\mathcal{F}_\xi,\mathbb{P})$ называется \textit{порожденным случайной величиной $\xi$}.
\end{defn}

\begin{defn}
    \textit{Распределение случайной величины} $\xi$~--- функция ${P_\xi: \mathfrak{B} \mapsto \mathbb{R}}$:
    \begin{equation*}
        P_\xi(B) = \myprob{\xi^{-1}(B)} = \myprob{\xi \in B}
    \end{equation*}
\end{defn}

\begin{rmrk}
    Распределение~--- это композиция отображений. Если ${\xi\colon \Omega \mapsto \mathbb{R}}$, то полный прообраз~--- это отображение ${\xi^{-1}\colon \mathfrak{B} \mapsto \mathcal{F}}$. В свою очередь, ${\mathbb{P}\colon \mathcal{F} \mapsto \mathbb{R}}$. Тогда
    \begin{equation*}
        P_{\xi} = \mathbb{P} \circ \xi^{-1}, \quad P_{\xi}\colon \mathfrak{B} \mapsto \mathbb{R}.
    \end{equation*}
\end{rmrk}

\hypertarget{induced_prob_space}{}
\begin{defn}
    Вероятностное пространство $(\mathbb{R}, \mathfrak{B}, P_\xi)$ называется \textit{индуцированным случайной величиной $\xi$}. 
        \footnote{{Внимательный читатель может задаться вопросом~--- а действительно ли тройки $\left(\Omega, \mathcal{F}_{\xi}, \mathbb{P}\right)$ и $\left(\mathbb{R}, \mathfrak{B}, P_{\xi}\right)$ удовлетворяют \hyperlink{prob_space}{\textit{определению вероятностного пространства}}}? 
        Короткий ответ~--- да, но я советую вам убедиться в этом самостоятельно 
        (особенно это касается проверки того, что распределение случайной величины является вероятностной мерой).}
\end{defn}

\subsubsection{Функция распределения, её свойства}
\begin{defn}
    \textit{Функция распределения} $F_\xi (x)$ случайной величины $\xi$~--- функция $F_\xi \colon \mathbb{R} \mapsto \mathbb{R}$:
    \begin{equation*}
        F_{\xi}(x) = P_{\xi}\left((-\infty, x)\right) = \mathbb{P}(\xi<x)
    \end{equation*}
\end{defn}

\begin{thm*}
    $F_\xi(x)$ однозначно определяет $P_\xi(B)$.
\end{thm*}
\begin{proof}
    Действительно, любое борелевское множество может быть представлено в виде разности числовой оси, одной или двух полупрямых и не более чем счётного объединения отрезков. 
    В силу однозначности определения $P_\xi([a;b]) = F_\xi(b + 0) - F_\xi(a)$ утверждение теоремы справедливо.
\end{proof}

\begin{namedthm}[Свойства функции распределения]\leavevmode
\begin{enumerate}
    \item $\forall \, x \quad 0 \leqslant F_\xi(x) \leqslant 1$;
    \item $x_1 < x_2 \; \Rightarrow \; F_\xi(x_1) \leqslant F_\xi(x_2) \quad \forall \, x_1, x_2$ (монотонно неубывает);
    \item $\lim\limits_{x \rightarrow +\infty} F_\xi(x) = 1, \lim\limits_{x \rightarrow -\infty} F_\xi(x) = 0$;
    \item $F_\xi(x_0 - 0) = \lim\limits_{x \rightarrow x_0 - 0}F_\xi(x) = F_\xi(x_0)$ (непрерывна слева).
\end{enumerate}
\end{namedthm}

\begin{proof}
\begin{enumerate}
    \item 
        Следует из свойств вероятности.
    \item 
        $x_1 < x_2 \Rightarrow \{\xi < x_1 \} \subseteq \{\xi < x_2\}$. 
        Из монотонности вероятности следует:
        \begin{equation*}
            F_\xi(x_1) = \myprob{\xi < x_1} \leqslant \myprob{\xi < x_2} = F_\xi(x_2).
        \end{equation*}
    \item 
        Пределы существуют в силу монотонности и ограниченности $F_\xi(x)$. 
        Докажем, что $F_\xi(-n) \xrightarrow[n \to +\infty]{} 0$.
        
        Рассмотрим последовательность вложенных событий 
        $B_n = \{\xi < -n \}$, 
        $B_{n+1} = \{\xi < -(n+1) \} \subseteq B_n = \{\xi < -n \} ~ \forall \, n ~ \geqslant 1$:
        \begin{equation*}
            \bigcap\limits_{j = 1}^{\infty}B_j = \{\omega \colon \xi(\omega) < x, \forall x \in \mathbb{R} \} \: \Rightarrow \: \bigcap\limits_{j = 1}^{\infty}B_j = \varnothing.
        \end{equation*}
    
        $F_\xi(-n) = \myprob{B_n} \xrightarrow[n \to +\infty]{} \myprob{B} = 0$ (в силу непрерывности вероятностной меры)
    
        Отсюда следует: $F_\xi(n) \xrightarrow[n \to +\infty]{} 1 \; \Leftrightarrow \; 1 - F_\xi(n) = \myprob{\xi \geqslant n} \xrightarrow[n \to +\infty]{} 0$.
% И что это за кусок доказательства в if?
\iffalse    
        Аналогично, пусть $B_n = \{\xi \geqslant n\}, B_{n+1} = \{\xi \geqslant (n+1) \} \subseteq B_n, \ldots$:
        \begin{equation*}
            \bigcap\limits_{j = 1}^{\infty}B_j = \varnothing \Rightarrow \xi(w) > x \quad \forall x \in \mathbb{R}
        \end{equation*}
        Следовательно, 
        \begin{equation*}
            1 - F_\xi(n) = \myprob{B_n} \xrightarrow[n \to +\infty]{} \myprob{B} = 0 \Rightarrow F_\xi(n) \xrightarrow[n \to +\infty]{}1
        \end{equation*}
\fi
    \item
        Достаточно показать, что $F_\xi(x_0 - 1/n) \xrightarrow[n \to +\infty]{} F_\xi(x_0)$. 
        Это равносильно 
        \begin{multline*}
            F_\xi(x_0) - F_\xi \left(x_0 - \frac{1}{n} \right) 
            = \myprob{\xi < x_0} - \mathbb{P}\left( \xi < x_0 - \frac{1}{n} \right) = \\ 
            = \mathbb{P}\left( x_0 - \frac{1}{n} \leqslant \xi < x_0 \right) \xrightarrow[n \to +\infty]{} 0.
        \end{multline*}
        Рассмотрим последовательность множеств $B_n = \{ x_0 - \frac{1}{n} \leqslant \xi < x_0 \} $. 
        $B_1 \supset B_2 \supset \ldots \supset B_n \supset \ldots$; 
        $\bigcap\limits_{k=1}^{n}B_k = B_n$;
        $\bigcap\limits_{k=1}^{\infty}B_k = \varnothing$.
        В силу непрерывности вероятностной меры 
        $\lim\limits_{n \to \infty} \mathbb{P}(B_n) = 
        \lim\limits_{n \to \infty} \mathbb{P}\left( \bigcap\limits_{k=1}^{n}B_k \right) = 
        \mathbb{P}(\varnothing) = 0$.
\end{enumerate}
\end{proof}

\begin{task}
Пусть есть не более чем счётное множество элементарных исходов $\Omega$. 
Рассмотрим функции ${f(\omega) \colon \Omega \mapsto \mathbb{R}}$.

Какая функция всегда измерима (т.е. измерима относительно любой ${\sigma \text{-алгебры}}$)? 
% Константа
Относительно какой ${\sigma \text{-алгебры}}$ измерима любая функция ${f(\omega) \colon \Omega \mapsto \mathbb{R}}$?
% Полной сигма-алгебры (2 ^ Omega)
\end{task}

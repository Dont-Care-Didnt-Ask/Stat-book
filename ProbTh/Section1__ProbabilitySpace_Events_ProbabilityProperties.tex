\section{Вероятностное пространство. Операции над событиями. Свойства вероятности}
\begin{defn}
    \textit{Пространство элементарных исходов} $\Omega$~--- любое непустое множество, содержащее все возможные результаты случайного эксперимента.
    Элементы $\omega \in \Omega$~--- \textit{элементарные исходы}.
\end{defn}

\begin{defn}
\textit{Алгебра} $\mathcal{A}$~--- множество подмножеств $\Omega$, обладающее следующими свойствами:

\begin{enumerate}
    \item 
        $\Omega \in \mathcal{A}$;
    \item 
        $A \in \mathcal{A} \: \Rightarrow \: \overline{A} \in \mathcal{A}$; \footnote{Здесь и в дальнейшем $\overline{A} \equiv \Omega \setminus A$~--- \textit{дополнение} к $A$.}
    \item 
        $A, B \in \mathcal{A} \: \Rightarrow \: A \cup B \in \mathcal{A}$ \\ 
        (по индукции: $A_1, A_2, \ldots, A_n \in \mathcal{A} \Rightarrow \bigcup\limits_{i=1}^n A_i \in \mathcal{A}$).
\end{enumerate}
\end{defn}

\begin{rmrk}
    Если $A, B \in \mathcal{A}$, то $A \cap B \equiv \overline{\overline{A} \cup \overline{B}} \in \mathcal{A}$.
\end{rmrk}

\begin{defn}
$\sigma \textit{{-алгебра~}} \mathcal{F}$~--- множество подмножеств $\Omega$, обладающее следующими свойствами:

\begin{enumerate}
    \item 
        $\Omega \in \mathcal{F}$;
    \item 
        $A \in \mathcal{F} \: \Rightarrow \: \overline{A} \in \mathcal{F}$;
    \item 
        $A_1, A_2,\ldots, A_n,\ldots \in \mathcal{F} \: \Rightarrow \: \bigcup\limits_{i=1}^\infty A_i \in \mathcal{F}$.
\end{enumerate}
\end{defn}
\begin{exmp}
    Множество всех подмножеств $2^{\Omega}$ и множество $\{\varnothing, \Omega\}$~--- $\sigma$-алгебры над $\Omega$.
\end{exmp}
\begin{rmrk}
    Любая $\sigma \text{-алгебра}$ является алгеброй. 
    Первые два пункта определений идентичны, рассмотрим третий. 
    Для любой конечной последовательности $A_1, A_2,\ldots, A_n \in \mathcal{A}$ составим соответствующую счётную последовательность $A_1, A_2, \ldots, A_n, A_{n+1}=\varnothing, A_{n+2}=\varnothing,\ldots \in \mathcal{A}$. 
    Пустое множество $\varnothing$ принадлежит $\sigma$-алгебре (т.к. $\varnothing = \overline{\Omega}$).
    По определению $\sigma \text{-алгебры}$: $\bigcup\limits_{i=1}^\infty A_i \in \mathcal{F} \Rightarrow \bigcup\limits_{i=1}^n A_i \in \mathcal{F}$, следовательно, выполнен третий пункт определения алгебры.
\end{rmrk}

\begin{defn}
    \textit{Случайное событие} $A$~--- элемент $\sigma \text{-алгебры~} \mathcal{F}$, т.е. некоторое подмножество элементарных исходов. 
    $A=\varnothing$~---\textit{невозможное событие}, $A=\Omega$~--- \textit{достоверное событие}. 
    Событие $\overline{A}$~--- \textit{противоположное} $A$, т.е. происходит тогда и только тогда, когда не происходит $A$.

Операции над событиями:

\begin{compactlist}
    \item 
        \textit{Объединение} $A \cup B$~--- происходит или $A$, или $B$, или оба вместе.
    \item 
        \textit{Пересечение} $A \cap B$ (или $AB$)~--- происходят и $A$ и $B$ вместе. Если $AB = \varnothing$, то события $A$ и $B$ называются \textit{несовместными}.
    \item 
        \textit{Разность} $A \setminus B$~--- происходит $A$ и не происходит $B$.
    \item 
        \textit{Симметрическая разность} $A \triangle B$~--- либо происходит $A$ и не происходит $B$, либо происходит $B$ и не происходит $A$.
\end{compactlist}
\end{defn}

\hypertarget{generated_sigma}{}
\begin{defn}
    $\sigma \text{-алгебра}$ \textit{порождена классом $K$}, если она является пересечением всех $\sigma \text{-алгебр}$, содержащих $K$, т.е. является \textit{минимальной $\sigma \text{-алгеброй}$}, содержащей $K$.
\end{defn}

\begin{exmp}
    Пусть $K = \{A\}$, тогда $\sigma (K) = \{\varnothing, A, \overline{A}, \Omega\}$.
\end{exmp}

\begin{defn}
    \textit{Вероятностная мера} или \textit{вероятность}~--- функция \\
    ${\mathbb{P}\colon \mathcal{F} \mapsto \mathbb{R}}$, обладающая следующими свойствами:

\begin{enumerate}
    \item 
        $\myprob{A} \geqslant 0 \quad \forall \, A \in \mathcal{F}$ (\textit{неотрицательность});
    \item 
        $\myprob{\Omega} = 1$ (\textit{нормировка});
    \item 
        $\forall \, A_1, A_2, \ldots, A_n, \ldots \in \mathcal{F},~ A_{i}A_{j} = \varnothing~ (i \ne j) \colon \mathbb{P}\left( \bigcup\limits_{i=1}^\infty A_i \right) = \sum\limits_{i=1}^\infty \myprob{A_i}$ \\
        (\textit{счётная аддитивность}).
\end{enumerate}
\end{defn}

\begin{rmrk}
    Из счётной аддитивности, очевидно, следует и конечная аддитивность, достаточно рассмотреть последовательность событий $A_1, A_2, \ldots, A_n, A_{n+1}=\varnothing, A_{n+2}=\varnothing,\ldots \in \mathcal{F}$.
\end{rmrk}
\pagebreak
\begin{namedthm}[Свойства вероятности]\leavevmode
    \begin{enumerate}
        \item 
            $\myprob{\varnothing}=0$;
        \item 
            $A, B \in \mathcal{F}, \, B \subset A \; \Rightarrow \; \myprob{A} \geqslant \myprob{B}$ (монотонность);
        \item 
            $\myprob{A \setminus B} = \myprob{A} - \myprob{AB}$;
        \item 
            $\myprob{A \cup B} = \myprob{A} + \myprob{B} - \myprob{AB}$;
        \item 
            $\forall A_1 \supseteq A_2 \supseteq \ldots \supseteq A_n \supseteq \ldots, \; \bigcap\limits_{n = 1}^{\infty} A_n = A \colon \lim\limits_{n \to \infty}\myprob{A_n} = \myprob{A}$ \\
            (непрерывность).
    \end{enumerate}
\end{namedthm}

\begin{proof}\leavevmode
    \begin{enumerate}
    \item 
        Рассмотрим последовательность событий $A_{1} = \Omega, A_{2} = \varnothing, \ldots, \\ 
        A_{n} = \varnothing, \ldots$ \, :
        \begin{equation*}
            \bigcup\limits_{i=1}^\infty A_i = \Omega \; \Rightarrow \; \mathbb{P} \left(\, \bigcup\limits_{i=1}^\infty A_i \right) = \myprob{\Omega} = 1.
        \end{equation*}
        При этом $A_{i}A_j = \varnothing~(i \ne j)$, следовательно, по пункту 3 определения вероятности: $\sum\limits_{i=2}^\infty \myprob{\varnothing} = 0 \: \Rightarrow \: \myprob{\varnothing} = 0$.
    \item 
        $B \subset A \; \Rightarrow \; A = (A \setminus B) \cup B$. 
        Из неотрицательности вероятности и того, что $(A \setminus B) \cap B = \varnothing$, следует, что $\myprob{A} = \myprob{A \setminus B} + \myprob{B} \geqslant \myprob{B}$. 
        Кроме того, в этом случае $\myprob{A \setminus B} = \myprob{A} - \myprob{B}$.
    \item 
        Доказательство аналогично пункту 2 при представлении $A$ в виде \\
        $A = (A \setminus B) \cup AB$.
    \item 
        Представим объединение событий $A$ и $B$ в виде $A \cup B = (A \setminus AB) \cup B$. 
        Очевидно, что $(A \setminus AB) \cap B = \varnothing$, откуда по пункту 3 определения вероятности следует:
        \begin{gather*}
            \myprob{A \cup B} = \myprob{(A \setminus AB) \cup B} = \myprob{A \setminus AB} + \myprob{B}, \\
            \myprob{A \setminus AB} = \myprob{A} - \myprob{A \cap AB} = \myprob{A} - \myprob{AB}.
        \end{gather*}
        Следовательно, $\myprob{A \cup B} = \myprob{A} + \myprob{B} - \myprob{AB}$.
    \item 
        Рассмотрим множества $C_n = A_n \setminus A_{n+1}, \, n \in \mathbb{N}$. 
        Они несовместны (пусть $l < m$, тогда $C_m = (A_m \setminus A_{m+1}) \subset A_m \subset A_{m-1} \subset \ldots \subset A_{l+1},$ но $C_l = A_l \setminus A_{l+1} \: \Rightarrow \: C_l \cap C_m = \varnothing$). 
        Тогда
        \begin{gather*}
            A_1 = \bigcup\limits_{n = 1}^{\infty} C_n \cup A, \quad \myprob{A_1} = \mathbb{P}\left(\, \bigcup\limits_{n = 1}^{\infty} C_n \cup A \right) = \sum\limits_{n=1}^{\infty} \myprob{C_n} + \myprob{A}, \\
            \sum\limits_{n=1}^{\infty} \myprob{C_n} = \myprob{A_1} - \myprob{A}.
        \end{gather*}
        Таким образом, ряд из вероятностей событий $C_n$ сходится. 
        Это равносильно тому, что его остаток $\sum\limits_{n=k}^{\infty} \myprob{C_n}$ стремится к нулю при $k \rightarrow \infty$. 
        Но при этом
        \begin{equation*}
            \bigcup\limits_{n = k}^{\infty} C_n \cup A = A_k, \quad \myprob{A_k} = \mathbb{P}\left(\, {\bigcup\limits_{n = k}^{\infty} C_n \cup A} \right) = \sum\limits_{n=k}^{\infty} \myprob{C_n} + \myprob{A}.
        \end{equation*}
        Перейдя в последнем равенстве к пределу при $k \rightarrow \infty$, получим
        $$ \lim\limits_{k \to \infty} \myprob{A_k} = \myprob{A}.$$
\end{enumerate}
\end{proof}

\begin{rmrk}
    Можно показать, что счётная аддитивность равносильна одновременному наличию конечной аддитивности и непрерывности. 
    Иными словами, третью аксиому в определении вероятности можно заменить на пару утверждений:
    \begin{itemize}
        \item 
            $\forall A, B, \: A \cap B = \varnothing \colon \Rightarrow \; \myprob{A \cup B} = \myprob{A} + \myprob{B}$ 
            (по индукции можно показать аддитивность для любого конечного $n$);
        \item 
            $\forall A_1 \supseteq A_2 \supseteq \ldots \supseteq A_n \supseteq \ldots; \bigcap\limits_{n = 1}^{\infty} A_n = A \colon \lim\limits_{n \to \infty}\myprob{A_n} = \myprob{A}$.
    \end{itemize}
\end{rmrk}

\hypertarget{prob_space}{}
\begin{defn}
    \textit{Вероятностное пространство}~--- тройка $(\Omega, \mathcal{F}, \mathbb{P})$, где $\Omega$~--- множество элементарных исходов, 
    $\mathcal{F}$~--- $\sigma$-алгебра над $\Omega$, вероятность $\mathbb{P}$ определена на $\mathcal{F}$.
\end{defn}
\begin{rmrk}
    Вероятностное пространство не является пространством в функциональном смысле.
\end{rmrk}

\begin{exmp}
    Тройка $([0, 1], \mathfrak{B}_{[0, 1]}, \lambda_{[0, 1]})$, где $\lambda$~--- мера Лебега, $\mathfrak{B}$~--- борелевская $\sigma$-алгебра, является вероятностным пространством. 
    В самом деле:
    \begin{compactlist}
        \item $\Omega = [0, 1]$~--- непустое множество;
        \item $\mathfrak{B}_{[0, 1]}$~--- $\sigma$-алгебра над $\Omega = [0, 1]$;
        \item $\lambda(\Omega) = \lambda\bigl( {[0, 1]} \bigr) = 1, \; \forall A \in \mathfrak{B}_{[0, 1]} \; \lambda(A) \geqslant 0$ и мера счётного объединения непересекающихся множеств есть счётная сумма их мер, т.е. выполняются три аксиомы вероятности.
    \end{compactlist}
    В то же время тройка $([-1, 1], \mathfrak{B}_{[-1, 1]}, \lambda_{[-1, 1]})$ не будет вероятностым пространством, т.к. нарушается свойство нормировки вероятностной меры: $\lambda(\Omega) = 2$, следовательно, вероятность не задана.
\end{exmp}

\begin{defn}
    Пусть $\Omega = \{\omega_1, \omega_2, \ldots, \omega_n\}$~--- конечное непустое множество, $\mathcal{F}$~--- множество всех подмножеств $\Omega$. 
    Положим $\mathbb{P} \bigl(\{\omega_i\}\bigr) = p_i$. 
    Вероятностное пространство, определённое таким образом,~--- \textit{дискретное вероятностное пространство}. 
    При этом для любого события $A = \{\omega_{i_1}, \ldots, \omega_{i_k}\}$ его вероятность $\myprob{A} = \sum\limits_{j=1}^k p_{i_j}$. 
    Таким образом вводится \textit{классическое определение вероятности} на дискретном вероятностном пространстве:
    \begin{equation*}
        p_1 = p_2 = \ldots =p_n=\cfrac{1}{n}, \quad \myprob{A} = \cfrac{k}{n},~ k=|A|,
    \end{equation*}
    т.е. все элементарные исходы считаются \textit{равновозможными}.
\end{defn}
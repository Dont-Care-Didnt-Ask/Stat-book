\section{Метод моментов. Свойства оценок, полученных методом моментов}

Пусть $\Sample$~--- выборка объёма $n$ из параметрического семейства распределений $\mathcal{P}_\theta$. 
Выберем функцию $g(y)\colon \mathbb{R} \mapsto \mathbb{R}$ так, чтобы существовал момент $\MExp g\left(X_{1}\right)=h(\theta)$ и функция $h(\theta)$ была обратима на $\Theta$. 
Разрешим полученное уравнение относительно $\theta$, а затем вместо истинного момента возьмём выборочный:

\begin{equation*}
    \theta = h^{-1}\bigl(\MExp g\left(X_{1}\right)\bigr), \quad 
    \theta^{*}=h^{-1}\bigl(\overline{g(\mathbf{X})}\bigr) = 
    h^{-1}\left(\frac{1}{n} \sum\limits_{i=1}^{n} g\left(X_{i}\right)\right)
\end{equation*}

Полученная оценка $\theta^{*}$~--- \textit{оценка метода моментов} для параметра $\theta$. 
Чаще всего берут $g(y)=y^{k}$. 
В этом случае, при условии обратимости функции~$h$ на~$\Omega$:
\begin{equation*}
    \MExp X_{1}^{k}=h(\theta), \quad 
    \theta = h^{-1}\left(\MExp X_{1}^{k}\right), \quad 
    \theta^{*} = h^{-1}\bigl( \overline{X^{k}} \bigr) = 
    h^{-1}\left(\frac{1}{n} \sum\limits_{i=1}^{n} X_{i}^{k}\right)
\end{equation*}

\begin{exmp}
    Рассмотрим равномерное распределение $\mathrm{U}[0, \theta]$. 
    Найдём оценку метода моментов для параметра $\theta$ по первому моменту:
    \begin{equation*}
        \MExp X_{1}=\frac{\theta}{2}, \quad 
        \theta = 2\,\MExp X_{1}, \quad 
        \theta_{1}^{*}=2 \overline{X}
    \end{equation*}
    Найдём оценку метода моментов k по $k$-му моменту:
    \begin{equation*}
        \MExp X_{1}^{k} =
        \int\limits_{0}^{\theta} y^{k} \frac{1}{\theta} \, dy = 
        \frac{\theta^{k}}{k+1}, \quad 
        \theta=\sqrt[k]{(k+1) \MExp X_{1}^{k}}, \quad 
        \theta_{k}^{*}=\sqrt[k]{(k+1) \overline{X^{k}}}
    \end{equation*}
\end{exmp}

\begin{thm*}
    Пусть $\theta^{*}=h^{-1}\bigl(\overline{g(\mathbf{X})}\bigr)$~--- оценка параметра $\theta$, полученная методом моментов, причём функция $h^{-1}$ непрерывна. 
    Тогда оценка $\theta^{*}$ состоятельна.
\end{thm*}

\begin{proof}
    По ЗБЧ Хинчина имеем:
    \begin{equation*}
        \overline{g(\mathbf{X})} = 
        \frac{1}{n} \sum\limits_{i=1}^{n} g\left(X_{i}\right) 
        \xrightarrow[n \to \infty]{\mathbb{P}} \MExp g\left(X_{1}\right) = 
        h(\theta)
    \end{equation*}

    Ввиду непрерывности функции $h^{-1}$:

    \begin{equation*}
        \theta^{*} = h^{-1}\bigl(\overline{g(\mathbf{X})}\bigr) 
        \xrightarrow[n \to \infty]{\mathbb{P}} h^{-1}\bigl(\MExp g\left(X_{1}\right)\bigr) = 
        h^{-1}\bigl(h(\theta)\bigr) = \theta.
    \end{equation*}
\end{proof}

\begin{defn}
    \textit{Асимптотически нормальная оценка} параметра $\theta$ с коэффициентом $\sigma^{2}(\theta)$~--- оценка $\theta^{*}$, т.ч. при $n \rightarrow \infty$ имеет место слабая сходимость к стандартному нормальному распределению: 
    \begin{equation*}
        \cfrac{\sqrt{n}(\theta^{*}-\theta)}{\sigma(\theta)} \xrightarrow[n \to \infty]{\text{d}} \mathbf{N}(0, 1).
    \end{equation*}
\end{defn}

\begin{lem}
    Пусть функция $g(y)$ такова, что $0 \neq \Var g\left(X_{1}\right)<\infty$. 
    Тогда статистика $\overline{g(\mathbf{X})}$ является асимптотически нормальной оценкой для $\MExp g\left(X_{1}\right)$ с коэффициентом $\sigma^{2}(\theta)=\Var g\left(X_{1}\right)$:

    \begin{equation*}
        \sqrt{n} \, \cfrac{\overline{g(\mathbf{X})}-\MExp g\left(X_{1}\right)}{\sqrt{\Var g\left(X_{1}\right)}} 
        \xrightarrow[n \to \infty]{\text{d}} \mathbf{N}(0, 1)
    \end{equation*}
\end{lem}

\begin{proof}
    Следует непосредственно из ЦПТ.
\end{proof}

\begin{rmrk}
    Следующая теорема утверждает асимптотическую нормальность оценок вида

    \begin{equation*}
        \theta^{*} = 
        H \bigl( \overline{g(\mathbf{X})} \bigr) = 
        H\left(\cfrac{g\left(X_{1}\right)+\ldots+g\left(X_{n}\right)}{n}\right)
    \end{equation*}
    которые обычно получаются при использовании метода моментов, при этом всегда $\theta = H \bigl(\MExp g\left(X_{1}\right) \bigr)$.
\end{rmrk}

\begin{thm*}
    Пусть функция $g(y)$ такова, что $0 \neq \Var g\left(X_{1}\right)<\infty$, функция $H(y)$ дифференцируема в точке $a=\MExp g\left(X_{1}\right)$ и её производная в этой точке $H^{\prime}(a)=\left.H^{\prime}(y)\right|_{y=a}$ отлична от нуля. 
    Тогда оценка $\theta^{*}=H\bigl(\overline{g(\mathbf{X})}\bigr)$ является асимптотически нормальной оценкой для параметра $\theta = H\bigl(\MExp g\left(X_{1}\right)\bigr) = H(a)$ с коэффициентом асимптотической нормальности $\sigma^{2}(\theta)=\bigl(H^{\prime}(a)\bigr)^{2} \cdot \Var g\left(X_{1}\right)$.
\end{thm*}

\begin{proof}
    Согласно ЗБЧ последовательность $\overline{g(\mathbf{X})}$ стремится к $a=\MExp g\left(X_{1}\right)$ по вероятности с ростом $n$: Функция

    \begin{equation*}
        G(y)=\left\{\begin{array}{ll}
        \cfrac{H(y)-H(a)}{y-a}, & y \neq a \\
        H^{\prime}(a), & y=a
        \end{array}\right.  
    \end{equation*}
    по условию непрерывна в точке $a$: 
    поскольку сходимость по вероятности сохраняется под действием непрерывной функции, получим,
    что $G\bigl(\overline{g(\mathbf{X})}\bigr) \xrightarrow[n \to \infty]{\mathbb{P}} G(a)=H^{\prime}(a)$.

    Заметим также, что по вышеприведённой лемме величина ${\sqrt{n}\bigl(\overline{g(\mathbf{X})} - a\bigr)}$ слабо сходится
    к нормальному распределению $\mathbf{N}\bigl(0, \Var g(X_{1})\bigr)$: 
    пусть $\xi$~--- случайная величина из этого распределения. Тогда

    \begin{equation*}
        \sqrt{n} \Bigl(H\bigl(\overline{g(\mathbf{X})}\bigr) - H(a)\Bigr) = 
        \sqrt{n} \bigl(\overline{g(\mathbf{X})} - a\bigr) \cdot G\bigl(\overline{g(\mathbf{X})}\bigr) 
        \xrightarrow[n \to \infty]{\text{d}} \xi \cdot H^{\prime}(a)
    \end{equation*}

    Мы использовали следующее свойство слабой сходимости: если $\xi_{n} \xrightarrow[n \to \infty]{\text{d}} \xi$ и $\eta_{n} \xrightarrow[n \to \infty]{\mathbb{P}} c=\mathrm{const}$, то $\xi_{n} \eta_{n} \xrightarrow[n \to \infty]{\text{d}} c \xi$.
    Но распределение случайной величины $\xi \cdot H^{\prime}(a)$ есть $\mathbf{N}\Bigl(0,\bigl(H^{\prime}(a)\bigr)^{2} \cdot \Var g(X_{1})\Bigr)$, откуда следует

    \begin{equation*}
        \sigma^{2}(\theta)=\bigl(H^{\prime}(a)\bigr)^{2} \cdot \Var g\left(X_{1}\right).
    \end{equation*}
\end{proof}

\subsection*{Лирическое вступление}
\addcontentsline{toc}{section}{Лирическое вступление}
В теории вероятностей, как правило, изучается поведение фиксированной вероятностной модели в зависимости от \textit{известных} параметров.
Имея модель, мы пытаемся предсказать, какие результаты будут возникать при проведении случайного эксперимента.

В математической статистике решается обратная задача~--- по имеющимся наблюдениям мы пытаемся восстановить (или построить) модель.
Для этого требуется определить\footnote{Может быть, точнее будет сказать "<выбрать">~--- ведь мы сами строим свою математическую модель.
Например, подбрасывая монетку, мы вольны учесть, что монетка может упасть на ребро, и рассматривать семейство распределений случайных величин с тремя значениями.}
семейство распределений, а затем конкретные значения параметров.

Данный курс математической статистики можно поделить на три раздела~--- точечное оценивание, интервальное оценивание и проверка гипотез.
Первые два посвящены поиску значений параметров при уже выбранном семействе распределений. 

\textit{Точечное оценивание}~--- это приближение значения неизвестного параметра отдельным числом ($\theta \approx 4$).

\textit{Интервальное оценивание}~--- это построение некоторого интервала (или, в случае многомерного параметра, области в $\mathbb{R}^m$), который содержит истинное значение параметра с вероятностью не ниже заданной ($\theta \in (3, 5)$ с вероятностью не меньше 0.95). 

\textit{Проверка гипотез}~--- это процесс определения того, противоречит ли рассматриваемая гипотеза имеющейся выборке данных (например, верно ли, что $\theta > 4$? 
Или что распределение наблюдаемой случайной величины~--- нормальное?).

Проверка гипотез может использоваться для выбора семейства распределений, для которых мы потом будем искать значения параметров.
Кроме того, многие методы точечного и интервального оценивания предполагают, что наблюдения независимы и одинаково распределены.
А эти утверждения, вообще говоря, также требуют проверки.

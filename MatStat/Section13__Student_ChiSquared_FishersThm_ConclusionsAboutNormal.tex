\section{Статистические выводы о параметрах нормального распределения. Распределения \texorpdfstring{$\chi^{2}$}{хи-квадрат} и Стьюдента. Теорема Фишера}

\begin{defn}
    Говорят, что случайная величина $\xi$ имеет \textit{гамма-распределение} с параметрами $\lambda > 0,~ \alpha > 0$ (или $\xi \sim \mathbf{\Gamma}(\lambda, \alpha)$), 
    если $\xi$ имеет следующую плотность распределения:
    \begin{equation*}
        p_{\xi}(x) = \begin{cases}
            0, & \text { если } x \leqslant 0 \\
            c \cdot x^{\alpha-1} e^{-\lambda x}, & \text { если } x>0
        \end{cases}
    \end{equation*}
    где постоянная $c$ вычисляется из свойства нормировки плотности:
    \begin{equation*}
        1 = 
        \int\limits_{-\infty}^{\infty} p_{\xi}(x) d x = 
        c \int\limits_{0}^{\infty} x^{\alpha-1} e^{-\lambda x} d x = 
        \frac{c}{\lambda^{\alpha}} \int\limits_{0}^{\infty}(\lambda x)^{\alpha-1} e^{-\lambda x} d(\lambda x) = 
        \frac{c}{\lambda^{\alpha}} \Gamma(\alpha),
    \end{equation*}
    откуда $c=\lambda^{\alpha} / \Gamma(\alpha)$.
\end{defn}

\begin{lem}
    Пусть $\xi_{1}, \ldots, \xi_{n}$ независимы, и $\xi_i \sim \mathbf{\Gamma}(\lambda, \alpha_i), i=\overline{1,n}$. 
    Тогда их сумма $S_{n}=\xi_{1}+\ldots+\xi_{n} \sim \mathbf{\Gamma}(\lambda, \alpha_1 + \ldots + \alpha_n)$.
\end{lem}
\begin{proof}
    Найдём для начала характеристическую функцию гамма-распределения:
    \begin{multline*}
        \int\limits_{-\infty}^{\infty} p_{\xi}(x) e^{itx} \, dx = 
        \int\limits_{-\infty}^{\infty} \frac{\lambda^\alpha \, x^{\alpha - 1}}{\Gamma(\alpha)} \, e^{-\lambda x}\, e^{itx} \,dx = 
        \left| \begin{array}{c}
            y  = -(\lambda - it)x \\
            dy = -(\lambda - it)dx
        \end{array} \right| = \\
        \int\limits_{-\infty}^{\infty} \frac{\lambda^\alpha \, y^{\alpha - 1}}{(\lambda - it)^{\alpha - 1} \, \Gamma(\alpha)} \, e^{-y} \frac{1}{\lambda - it} \,dy = 
        \frac{\lambda^\alpha}{(\lambda - it)^{\alpha}} \int\limits_{-\infty}^{\infty} \frac{y^{\alpha - 1}}{\Gamma(\alpha)} \, e^{-y} \, dy = 
        \left(\frac{1}{1 - \frac{it}{\lambda}} \right)^\alpha \!.
    \end{multline*}
    Теперь вспомним, что если $\xi, \eta$ независимы, то характеристическая функция $f_{\xi + \eta}(t) = f_{\xi}(t) f_{\eta}(t)$.
    Тогда, если у нас есть независимые случайные величины $\xi_1, \ldots, \xi_n, \: \xi_i \sim \Gamma(\lambda, \alpha_i)$, 
    то характеристическая функция суммы~---
    \begin{equation*}
        \left(\frac{1}{1 - \frac{it}{\lambda}} \right)^\alpha_1 \cdot \ldots \cdot \left(\frac{1}{1 - \frac{it}{\lambda}} \right)^\alpha_n = 
        \left(\frac{1}{1 - \frac{it}{\lambda}} \right)^{\alpha_1 + \ldots + \alpha_n}
    \end{equation*}

    По теореме Леви о непрерывности заключаем, что сумма имеет распределение $\Gamma(\lambda, \alpha_1 + \ldots + \alpha_n)$.
\end{proof}

\begin{lem}
    Если $\xi \sim \mathbf{N}(0,1)$, то $\xi^2 \sim \mathbf{\Gamma}(1/2, 1/2)$.
\end{lem}

\begin{proof}
    Найдём функцию распределения случайной величины~${\eta = \xi^2}$:
    \begin{gather*}
        \mathbb{P}(\xi^2 < x) = \mathbb{P}(|\xi| < \sqrt{x}) = 2 \, \mathbb{P}(0 < x < \sqrt{x}) = \\
        2 \int\limits_{0}^{\sqrt{x}} \frac{1}{\sqrt{2 \pi}}\,  e^{u^2 /2} \,du = 
        \left| \begin{array}{cc}
            z = u^2    & u = \sqrt{z} \\
            dz = 2u du & du = \frac{dz}{2 \sqrt{z}} \\
            u = 0 \Rightarrow z = 0 & u = \sqrt{x} \Rightarrow z = x
        \end{array} \right| = \\
        2 \int\limits_{0}^{x} \frac{1}{\sqrt{2 \pi}} \, e^{-\frac{1}{2}z} \frac{dz}{2 \sqrt{z}} = 
        \int\limits_{0}^{x} \frac{\left(\frac{1}{2}\right)^{\frac{1}{2}} z^{\frac{1}{2} - 1}}{\sqrt{\pi}} e^{-\frac{1}{2}z} dz.
    \end{gather*}
    Но это в точности функция гамма-распределения с параметрами $\lambda = \frac{1}{2}, \alpha = \frac{1}{2}$ (ведь $\Gamma\left(\frac{1}{2}\right) = \sqrt{\pi}$).
\end{proof}

\begin{crlr}
    Если $\xi_{1}, \ldots, \xi_{k}$ независимы и $\xi_i \sim \mathbf{N}(0,1)$, то случайная величина $\chi^{2}=\xi_{1}^{2}+\ldots+\xi_{k}^{2} \sim \mathbf{\Gamma}(1/2, k/2)$
\end{crlr}

\begin{defn}
    Распределение суммы $k$ квадратов независимых случайных величин со стандартным нормальным распределением называется \textit{распределением хи-квадрат} с $k$ степенями свободы (обозначение: $\chi^2(k)$).
\end{defn}
Плотность распределения $\chi^2(k)$ имеет вид
\begin{equation*}
    f(y) = \begin{cases}
        \cfrac{1}{2^{x / 2} \Gamma(k / 2)} \, x^{\frac{k}{2}-1} e^{-x / 2}, & \text { если } x>0 \\
        0, & \text { если } x \leqslant 0
    \end{cases}
\end{equation*}
\begin{rmrk}
    $\chi^2(2) = \mathbf{\Gamma}(1/2,1) = \mathbf{Exp}(1/2)$.
\end{rmrk}

\begin{namedthm}[Свойства распределения $\chi^{2}$]\leavevmode
\begin{enumerate}
    \item Если случайные величины $\xi_1 \sim \chi^{2}(k)$ и $\xi_2 \sim \chi^{2}(m)$ независимы, то их сумма $\xi_1+\xi_1 \sim \chi^{2}(k+m)$;
    \item $\mathbb{E} \chi^{2}=k, \quad \mathbb{D} \chi^{2}=2 k$.
    \item Пусть дана последовательность случайных величин $\chi_{n}^{2}$. Тогда при $n \to \infty$:
    \begin{equation*}
        \frac{\chi_{n}^{2}}{n} \xrightarrow[n \to \infty]{\mathbb{P}} 1, 
        \quad \frac{\chi_{n}^{2}-n}{\sqrt{2 n}} \xrightarrow[n \to \infty]{\text{d}} \mathbf{N}(0,1)
    \end{equation*}
    \item Пусть случайные величины $\xi_1, \ldots, \xi_n$ независимы и $\xi_i \sim \mathbf{N}(a,\sigma^{2})$. Тогда
    \begin{equation*}
        \sum\limits_{i=1}^{k}\left(\frac{\xi_{i}-a}{\sigma}\right)^{2} \sim \chi^{2}(k).
    \end{equation*}
\end{enumerate}
\end{namedthm}

\begin{defn}
    Пусть $\xi_{0}, \xi_{1}, \ldots, \xi_{k}$ независимы и $\xi_i \sim \mathbf{N}(0,1)$. Распределение случайной величины
    \begin{equation*}
        t_{k}
        = \frac{\xi_{0}}{\sqrt{\frac{\xi_{1}^{2} + \ldots + \xi_{k}^{2}}{k}}} 
        = \frac{\xi_0}{\sqrt{x_{k}^{2} / k}}
    \end{equation*}
    называется \textit{распределением Стьюдента ($t$-распределением} с $k$ степенями свободы ($St(k)$ или $\mathbf{T}(k)$).
\end{defn}
Плотность распределения $\mathbf{T}(k)$ имеет вид
\begin{equation*}
    f_{k}(y)=\frac{\Gamma\bigl((k+1) / 2\bigr)}{\sqrt{\pi k} \, \Gamma(k / 2)}\left(1+\frac{y^{2}}{k}\right)^{-(k+1) / 2}
\end{equation*}

\begin{namedthm}[Свойства распределения Стьюдента]\leavevmode
\begin{enumerate}
    \item Распределение Стьюдента симметрично, т.е. если $t_k \sim \mathbf{T}(k)$, то $-t_k \sim \mathbf{T}(k)$.
    \item $\mathbf{T}(k) \xrightarrow[k \to \infty]{} \mathbf{N}(0,1)$.
    \item У распределения Стьюдента $\mathbf{T}(k)$ существуют только моменты порядка $m < k$, при этом все существующие моменты нечётного порядка равны нулю.
\end{enumerate}
\end{namedthm}

\begin{namedthm}[Теорема Фишера]
Пусть случайные величины $X_1, \ldots, X_n$ независимы и ${X_i \sim \mathbf{N}(a,\sigma^{2})}$. Тогда:
\begin{enumerate}
    \item $\sqrt{n} \frac{\overline{X}-a}{\sigma} \sim \mathrm{N}(0,1)$
    \item $\frac{(n-1) S_{0}^{2}}{\sigma^{2}}=\sum\limits_{i=1}^{n} \frac{\left(X_{i}-\overline{X}\right)^{2}}{\sigma^{2}} \sim \chi^{2}(n-1)$
    \item Случайные величины $\overline{X}$ и $S_{0}^{2}$ независимы.
\end{enumerate}
\end{namedthm}
\begin{crlr}
    Пусть случайные величины $X_1, \ldots, X_n$ независимы и ${X_i \sim \mathbf{N}(a,\sigma^{2})}$. Тогда:
    \begin{enumerate}
        \item $\sqrt{n} \frac{\overline{X}-a}{\sigma} \sim \mathrm{N}(0,1)$ (для $a$ при известном $\sigma^{2}$)
        \item $\sum\limits_{i=1}^{n}\left(\frac{X_{i}-a}{\sigma}\right)^{2} \sim \chi^{2}(n)$ (для $\sigma^{2}$ при известном $a$)
        \item $\frac{(n-1) S_{0}^{2}}{\sigma^{2}} \sim \chi^{2}(n-1)$ (для $\sigma^{2}$ при неизвестном $a$)
        \item $\sqrt{n} \frac{\overline{X}-a}{S_{0}} \sim \mathrm{T}(n-1)$ (для $a$ при неизвестном $\sigma^{2}$)
    \end{enumerate}
\end{crlr}

\subsubsection{Статистические выводы о параметрах нормального распределения}

Пусть $X_{1}, \ldots, X_{n}$~--- выборка объёма $n$ из распределения $\mathrm{N}_{a, \sigma^{2}}$. Построим точные доверительные интервалы (ДИ) с уровнем доверия $\alpha$ для параметров нормального распределения, используя следствие из теоремы Фишера.
\begin{enumerate}
    \item ДИ для $a$ при известном $\sigma^{2}$:
    \begin{equation*}
        \mathbb{P}\left(\overline{X}-\frac{\tau \sigma}{\sqrt{n}}<a<\overline{X}+\frac{\tau \sigma}{\sqrt{n}}\right)=\alpha, ~ \text {где} ~ \varphi_{0,1}(\tau)=\frac{1 + \alpha}{2}
    \end{equation*}
    \item ДИ для $\sigma^{2}$ при известном $a$:
    \begin{equation*}
        \frac{n S_{1}^{2}}{\sigma^{2}} \sim \chi^{2}(n),~ \text {где}~ S_{1}^{2}=\frac{1}{n} \sum\limits_{i=1}^{n}\left(X_{i}-a\right)^{2}
    \end{equation*}
    Пусть $g_1$ и $g_2$~--- квантили распределения $\chi^{2}(n)$ уровней $\frac{1-\alpha}{2}$ и $\frac{1+\alpha}{2}$ соответственно. Тогда
    \begin{equation*}
        \frac{(n-1) S_{0}^{2}}{\sigma^{2}} \sim \chi^{2}(n-1),~ \text {где}~ S_{0}^{2}=\frac{1}{n-1} \sum\limits_{i=1}^{n}\left(X_{i}-\overline{X}\right)^{2}
    \end{equation*}
    \item ДИ для $\sigma^{2}$ при неизвестном $a$:
    \begin{equation*}
        \frac{(n-1) S_{0}^{2}}{\sigma^{2}} \sim \chi^{2}(n-1),~ \text {где}~ S_{0}^{2}=\frac{1}{n-1} \sum\limits_{i=1}^{n}\left(X_{i}-\overline{X}\right)^{2}
    \end{equation*}
    Пусть $g_1$ и $g_2$~--- квантили распределения $\chi^{2}(n-1)$ уровней $\frac{1-\alpha}{2}$ и $\frac{1+\alpha}{2}$ соответственно. Тогда
    \begin{equation*}
        \alpha=\mathbb{P}\left(g_{1}<\frac{(n-1) S_{0}^{2}}{\sigma^{2}}<g_{2}\right)=\mathbb{P}\left(\frac{(n-1) S_{0}^{2}}{g_{2}}<\sigma^{2}<\frac{(n-1) S_{0}^{2}}{g_{1}}\right)
    \end{equation*}
    \item ДИ для $a$ при неизвестном $\sigma^{2}$:
    \begin{equation*}
        \sqrt{n} \frac{\overline{X}-a}{S_{0}} \sim \mathrm{T}(n-1)
    \end{equation*}
    Пусть $c$~--- квантиль распределения $\mathrm{T}(n-1)$ уровня $\frac{1-\alpha}{2}$. Распределение Стьюдента симметрично, поэтому
    \begin{equation*}
        \alpha=\mathbb{P}\left(-c<\sqrt{n} \frac{\overline{X}-a}{S_{0}}<c\right)=\mathbb{P}\left(\overline{X}-\frac{c S_{0}}{\sqrt{n}}<a<\overline{X}+\frac{c S_{0}}{\sqrt{n}}\right)
    \end{equation*}
\end{enumerate}

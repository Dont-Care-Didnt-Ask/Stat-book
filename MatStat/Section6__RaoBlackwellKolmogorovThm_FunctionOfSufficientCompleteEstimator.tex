\section{Теорема Рао"--~Блекуэлла"--~Колмогорова. Оптимальность оценок, являющихся функцией полной достаточной статистики}

\begin{namedthm}[Теорема Рао"--~Блекуэлла"--~Колмогорова] 
    Если существует оптимальная оценка для $\tau(\theta)$, то она является функцией от достаточной статистики.
\end{namedthm}

\begin{proof}
    В доказательстве используются следующие свойства условного математического ожидания: 
    \begin{gather*}
        \mathbb{E} f(x, z)=\mathbb{E}\Bigl(\mathbb{E}\bigl( f(x, z) | z\bigr) \Bigr), \\
        \mathbb{E}\bigl(g(z) | z\bigr) = g(z).
    \end{gather*}

    Мы докажем даже более сильное утверждение: для любой несмещённой оценки мы можем построить новую оценку, являющуюся функцией от достаточной статистики, при этом дисперсия построенной оценки будет не больше исходной.
    Отсюда вытекает и утверждение теоремы~--- ведь оптимальная оценка является несмещённой, соответственно, мы можем построить новую оценку, которая будет равномерно по $\theta$ не хуже оптимальной.
    Но оптимальная оценка единственна, а значит, она сама является функцией от достаточной статистики, что и требуется доказать.

    \begin{enumerate}
        \item 
            Построим искомую оценку.
            Пусть $T(\mathbf{X})$~--- достаточная статистика, $T_1(\mathbf{X})$~--- несмещённая оценка $\tau(\theta)$, т.е. $\MExp T_{1}(\mathbf{X})=\tau(\theta)$. 
            Рассмотрим функцию $H(T)=\MExp\left(T_{1} | T\right)$. 
            Тогда из первого свойства следует:
            \begin{multline*}
                \MExp H(T)=\MExp\bigl(\MExp\left(T_{1} | T\right)\bigr) = 
                \MExp T_{1}=\tau(\theta) \Rightarrow \\ H(T) \text{~--- несмещённая оценка~} \tau(\theta).
            \end{multline*}

        \item 
            Покажем, что её дисперсия не превосходит дисперсию исходной:
            \begin{gather*}
                \Var T_1 = \MExp \bigl(T_1 - \tau(\theta)\bigr)^2 = \\
                \MExp \Bigl(T_1 - H(T) + H(T) - \tau(\theta) \Bigr)^2 = \\
                \underbrace{\MExp \bigl(T_1 - H(T) \bigr)^2}_{\geqslant 0} + \underbrace{2 \, \MExp \Bigl[\bigl(T_1 - H(T)\bigr) \bigl(H(T) - \tau(\theta)\bigr)\Bigr]}_{?} + \, \Var H(T)
            \end{gather*}
            
            Оценим второе слагаемое, пользуясь тем, что $H(T)$~--- функция от $T$ и может быть вынесена из под условного математического ожидания как константа:
            \begin{gather*}
                \MExp \Bigl[ \bigl((T_{1}-H(T)\bigr) {\color{red}\bigl(} H(T) - \tau(\theta) {\color{red}\bigr)} \Bigr] = \\
                = \MExp\biggl( \MExp\Bigl[ \bigl(T_{1}-H(T)\bigr) {\color{red}\bigl(} H(T)-\tau(\theta) {\color{red}\bigr)} | \, T \Bigr] \biggr) = \\
                = \MExp\biggl( {\color{red}\bigl(} H(T)-\tau(\theta) {\color{red}\bigr)} \, \MExp \Bigl[ \bigl(T_{1} - H(T)\bigr) | \, T \Bigr] \biggr) = \\
                = \MExp\biggl( {\color{red}\bigl(} H(T)-\tau(\theta) {\color{red}\bigr)} \Bigl[ \MExp \bigl(T_{1} | T\bigr) - \MExp \bigl(H(T) | \, T \bigr) \Bigr] \biggr) = \\
                = \MExp\biggl( {\color{red}\bigl(} H(T)-\tau(\theta) {\color{red}\bigr)} \Bigl[ H(T) - H(T) \Bigr] \biggr) = 0.
            \end{gather*}

            Отсюда и вытекает, что
            \begin{equation*}
                \Var T_{1} = \underbrace{\MExp \bigl(T_1 - H(T)\bigr)^2}_{\geqslant 0} + \Var H(T) \; \geqslant \; \Var H(T).
            \end{equation*}
    \end{enumerate}
    Таким образом, если существует оптимальная оценка $T_1$, то $H(T)$ тоже оптимальна, но мы знаем, что оптимальная оценка единственна.
    Осталось заметить, что $H(T) = \MExp \bigl( T_1 | T\bigr) \equiv f(T)$~--- функция от достаточной статистики.
\end{proof}

\begin{namedthm}[Теорема Колмогорова]
    %Если $T(\mathbf{X})$~--- полная достаточная статистика, то она является оптимальной оценкой своего математического ожидания. (Можно сформулировать более сильный вариант)
    Пусть $T(\mathbf{X})$~--- полная достаточная статистика.
    Пусть $\varphi(x)$~--- борелевская функция, и $\MExp \varphi\bigl(T(\mathbf{X})\bigr) = \tau(\theta)$.
    Тогда $\varphi\bigl(T(\mathbf{X})\bigr)$~--- оптимальная оценка для $\tau(\theta)$.
\end{namedthm}

\begin{proof}
    Из теоремы Рао"--~Блекуэлла"--~Колмогорова следует, что если существует несмещённая оценка для $\tau(\theta)$, 
    то существует также и несмещённая оценка $\tau(\theta)$, являющаяся функцией от достаточной статистики.

    Пусть $T_1 = \varphi_1(T)$~--- несмещённая оценка $\tau(\theta)$ и $T$~--- ПДС (полная достаточная статистика).
    Допустим, что $\exists$ ещё одна несмещённая оценка $T_2 = \varphi_2(T)$.
    Тогда $\Always$
    \begin{equation*}
        \MExp \Bigl(\varphi_1\bigl(T(\mathbf{X})\bigr) - \varphi_2\bigl(T(\mathbf{X})\bigr) \Bigr) = 0.
    \end{equation*}
    Но $T(\mathbf{X})$ полная, $\Rightarrow \; \varphi_1(y) \overset{\text{п.н.}}{=} \varphi_2(y)$ на области значений $T(\mathbf{X})$.
    Таким образом, несмещённая оценка, являющаяся функцией от достаточной статистики, единственна.
    Значит, она и будет оптимальной оценкой $\tau(\theta)$.

    %Докажем, что $T(\mathbf{X})$ является единственной несмещённой оценкой для $\MExp T(\mathbf{X})$. (Несмещённых оценок много, единственна несмещённая оценка--функция от ПДС)
    %Тогда $T(\mathbf{X})$ будет оптимальной оценкой. 
    %Предположим, что $T_1(\mathbf{X})$~--- оптимальная оценка для $\MExp T(\mathbf{X})$. 
    %Из теоремы Рао"--~Блекуэлла"--~Колмогорова получаем, что $T_{1}=H(T)$ и $\MExp T_{1}=\MExp T$. 
    %Тогда:
    %\begin{equation*}
    %    \MExp \underbrace{\Bigl(T(\mathbf{X})-H\bigl(T(\mathbf{X})\bigr)\Bigr)}_{\varphi(T)}=0
    %\end{equation*}
    %
    %Из условия полноты $T(\mathbf{X})$ следует, что $\varphi(T)=0$ с вероятностью 1, т.е. $T=H(T)$ с вероятностью 1.
\end{proof}
